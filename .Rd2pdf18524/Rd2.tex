\documentclass[a4paper]{book}
\usepackage[times,inconsolata,hyper]{Rd}
\usepackage{makeidx}
\usepackage[utf8]{inputenc} % @SET ENCODING@
% \usepackage{graphicx} % @USE GRAPHICX@
\makeindex{}
\begin{document}
\chapter*{}
\begin{center}
{\textbf{\huge Package}}
\par\bigskip{\large \today}
\end{center}
\begin{description}
\raggedright{}
\inputencoding{utf8}
\item[Type]\AsIs{Package}
\item[Title]\AsIs{What the Package Does (Title Case)}
\item[Version]\AsIs{0.1.0}
\item[Author]\AsIs{Who wrote it}
\item[Maintainer]\AsIs{The package maintainer }\email{yourself@somewhere.net}\AsIs{}
\item[Description]\AsIs{More about what it does (maybe more than one line)
Use four spaces when indenting paragraphs within the Description.}
\item[License]\AsIs{What license is it under?}
\item[Encoding]\AsIs{UTF-8}
\item[LazyData]\AsIs{true}
\end{description}
\Rdcontents{\R{} topics documented:}
\inputencoding{utf8}
\HeaderA{ZILGM-package}{A R Package to estimate Local Markov Network for Count data with zero-inflated and overdispersion.}{ZILGM.Rdash.package}
%
\begin{Description}\relax
A R Package to estimate Local Markov Network for Count data with zero-inflated and overdispersion.
\end{Description}
%
\begin{Details}\relax

\Tabular{rl}{
Package: & ZILGM\\{}
Type: & Package\\{}
Version: & 1.0\\{}
Date: & 2019-03-25\\{}
License: & GPL-2\\{}
}
\end{Details}
%
\begin{Author}\relax
Park Beomjin, 

Maintainer : Park Beomjin <bbbeomjin@gmail.com>
\end{Author}
%
\begin{References}\relax
Network analysis for Count data with excess zero
\end{References}
%
\begin{Examples}
\begin{ExampleCode}
library(devtools)
install_github(bbeomjin/ZILGM)
library(ZILGM)
\end{ExampleCode}
\end{Examples}
\inputencoding{utf8}
\HeaderA{find\_lammax}{Compute the maximum lambda}{find.Rul.lammax}
%
\begin{Description}\relax
Compute the maximum lambda
\end{Description}
%
\begin{Usage}
\begin{verbatim}
find_lammax(X)
\end{verbatim}
\end{Usage}
%
\begin{Arguments}
\begin{ldescription}
\item[\code{X}] 
A \emph{n} x \emph{p} data matrix.

\end{ldescription}
\end{Arguments}
%
\begin{Details}\relax
Zero-inflated local graphical model
\end{Details}
%
\begin{Value}
An S3 object with the following slots
\begin{ldescription}
\item[\code{lammax}] a maximum lambda from data matrix
\end{ldescription}
\end{Value}
%
\begin{Author}\relax
Park Beomjin, 
\end{Author}
%
\begin{References}\relax
Network analysis for Count data with excess zero
\end{References}
%
\begin{Examples}
\begin{ExampleCode}
Not yet
\end{ExampleCode}
\end{Examples}
\inputencoding{utf8}
\HeaderA{network\_gen}{Generate random graph structure}{network.Rul.gen}
%
\begin{Description}\relax
Generate random graph structure
\end{Description}
%
\begin{Usage}
\begin{verbatim}
generate_network(node, prob)
\end{verbatim}
\end{Usage}
%
\begin{Arguments}
\begin{ldescription}
\item[\code{node}] 
The number of nodes to generate graph.

\item[\code{prob}] 
The probability of appearing edges. 

\end{ldescription}
\end{Arguments}
%
\begin{Details}\relax
For simulated data
\end{Details}
%
\begin{Value}
\begin{ldescription}
\item[\code{networkmat}] Adjacency matrix for random graph.
\end{ldescription}
\end{Value}
%
\begin{Author}\relax
Park Beomjin,
\end{Author}
%
\begin{References}\relax
Network analysis for Count data with excess zeros
\end{References}
%
\begin{SeeAlso}\relax
\LinkA{zilgm}{zilgm}
\end{SeeAlso}
%
\begin{Examples}
\begin{ExampleCode}
require(ZILGM)
\end{ExampleCode}
\end{Examples}
\inputencoding{utf8}
\HeaderA{zilgm}{Zero-inflated Local Graphical Model}{zilgm}
%
\begin{Description}\relax
Zero-inflated Local Graphical Model
\end{Description}
%
\begin{Usage}
\begin{verbatim}
zilgm(X, lambda = NULL, nlambda = 50, family = c("Poisson", "NBI", "NBII"), update_type = c("IRLS", "MM"), 
                sym = c("AND", "OR"), thresh = 1e-6, weights_mat = NULL, penalty_mat = NULL,
                do_boot = FALSE, boot_num = 10, beta = 0.05, lambda_min_ratio = 1e-4,
                init_select = FALSE, nCores, ...)
\end{verbatim}
\end{Usage}
%
\begin{Arguments}
\begin{ldescription}
\item[\code{X}] 
A \emph{n} x \emph{p} data matrix, where \emph{n} is the number of observations and \emph{p} is the number of variables or nodes.

\item[\code{lambda}] 
A sequence of regularization parameter to control a level of \emph{l\_1}-penalty.

\item[\code{nlambda}] 
The number of regularization parameter.

\item[\code{family}] 
Types of node-conditional distribution to be assumed among zero-inflated distributions.

\item[\code{update\_type}] 
Algorithm for estimating edge coefficients.

\item[\code{sym}] 
Symmetrize the output graphs. If \code{sym = "AND"}, the edge between node \emph{i} and node \emph{j} is selected only when both node \emph{i} and node \emph{j} are selected as neighbors for each other. If \code{sym = "OR"}, 
the edge is selected when either node \emph{i} or node \emph{j} is selected as the neighbor for each other.

\item[\code{thresh}] 
Threshold value for the estimated edge coefficients.

\item[\code{weights\_mat}] 
A \emph{n} x \emph{p} matrix containing weights for observations in each node

\item[\code{penalty\_mat}] 
A \emph{p} x \emph{p} matrix containing weights for each edge coefficients

\item[\code{do\_boot}] 
A logical. Whether to use Stability Apprach to Regularization Selection (StARS).

\item[\code{boot\_num}] 
The number of iteration for StARS.

\item[\code{beta}] 
Threshold value on sparsity of the network.

\item[\code{lambda\_min\_ratio}] 
The smallest value for lambda, as a fraction of the \emph{lambda\textasciicircum{}max} of the regularization parameter.

\item[\code{init\_select}] 
A logical. Whether to use initial selection.

\item[\code{nCores}] 
The number of cores to use for parallel computing.

\item[\code{...}] 
Other arguments that can be passed to neighborhood selection function(\code{zilgm\_poisson, zilgm\_negbin, and zilgm\_negbin2})

\end{ldescription}
\end{Arguments}
%
\begin{Details}\relax
Zero-inflated local graphical model
\end{Details}
%
\begin{Value}
An S3 object with the following slots
\begin{ldescription}
\item[\code{network}] a list of \emph{p} x \emph{p} matrices of estimated networks along the regularization path.
\item[\code{coef\_network}] a array of \emph{p} x \emph{p} x nlambda of estimated edge coefficients matrix along the regularization path.
\item[\code{lambda}] vector used for regularization path.
\item[\code{v}] vector of network variability measured for each regularization level.
\item[\code{opt\_lambda}] The lambda that gives the optimal network.
\end{ldescription}
\end{Value}
%
\begin{Author}\relax
Park Beomjin, 
\end{Author}
%
\begin{References}\relax
Network analysis for Count data with excess zero
\end{References}
%
\begin{Examples}
\begin{ExampleCode}
Not yet
\end{ExampleCode}
\end{Examples}
\inputencoding{utf8}
\HeaderA{zilgm\_negbin}{Neiborhood selection under the zero-inflated negative binomial distribution for Zero-inflated Local Graphical Model}{zilgm.Rul.negbin}
%
\begin{Description}\relax
Zero-inflated negative binomial regression with \emph{l\_1}-regularization.
\end{Description}
%
\begin{Usage}
\begin{verbatim}
zilgm_negbin(y, x, lambda, weights = NULL, update_type = c("IRLS", "MM"), penalty.factor = NULL,
              thresh = 1e-6, EM_tol = 1e-6, EM_iter = 500, tol = 1e-6, maxit = 1e+3, fixed_theta = FALSE)
\end{verbatim}
\end{Usage}
%
\begin{Arguments}
\begin{ldescription}
\item[\code{y}] 
A response \emph{y}.

\item[\code{x}] 
A design matrix \emph{x}.

\item[\code{lambda}] 
A regularization parameter to control a level of \emph{l\_1}-penalty.

\item[\code{weights}] 
Weights vector for observations

\item[\code{update\_type}] 
Algorithm for estimating coefficients

\item[\code{penalty.factor}] 
Weights vector for coefficients of each variable.

\item[\code{thresh}] 
Threshold value for the estimated coefficients.

\item[\code{EM\_tol}] 
Convergence tolerance for EM algorithm.

\item[\code{EM\_iter}] 
Maximum number of EM algorithm iterations.

\item[\code{tol}] 
Convergence tolerance for coordinate descent.

\item[\code{maxit}] 
Maximum number of coordinate descent iterations.

\item[\code{fixed\_theta}] 
A logical. Whether to estimate dispersion parameter theta.

\end{ldescription}
\end{Arguments}
%
\begin{Details}\relax
Zero-inflated local graphical model
\end{Details}
%
\begin{Value}
An S3 object with the following slots
\begin{ldescription}
\item[\code{beta}] Estimated coefficients vector.
\item[\code{theta}] Estimated dispersion parameter theta.
\item[\code{prob}] Estimated probability of structural zero.
\item[\code{pos\_zero}] Indices of zero values.
\item[\code{iteration}] Iteration numbers until convergence.
\item[\code{loglik}] \emph{l\_1}-penalized negative log-likelihood value.
\item[\code{call}] The mathced call.
\end{ldescription}
\end{Value}
%
\begin{Author}\relax
Park Beomjin, 
\end{Author}
%
\begin{References}\relax
Network analysis for Count data with excess zero
\end{References}
\inputencoding{utf8}
\HeaderA{zilgm\_negbin2}{Neiborhood selection under the zero-inflated negative binomial II distribution for Zero-inflated Local Graphical Model}{zilgm.Rul.negbin2}
%
\begin{Description}\relax
Zero-inflated negative binomial regression with \emph{l\_1}-regularization.
\end{Description}
%
\begin{Usage}
\begin{verbatim}
zilgm_negbin2(y, x, lambda, weights = NULL, update_type = c("IRLS", "MM"), penalty.factor = NULL,
              thresh = 1e-6, EM_tol = 1e-6, EM_iter = 500, tol = 1e-6, maxit = 1e+3, fixed_sigma = FALSE)
\end{verbatim}
\end{Usage}
%
\begin{Arguments}
\begin{ldescription}
\item[\code{y}] 
A response \emph{y}.

\item[\code{x}] 
A design matrix \emph{x}.

\item[\code{lambda}] 
A regularization parameter to control a level of \emph{l\_1}-penalty.

\item[\code{weights}] 
Weights vector for observations

\item[\code{update\_type}] 
Algorithm for estimating coefficients

\item[\code{penalty.factor}] 
Weights vector for coefficients of each variable.

\item[\code{thresh}] 
Threshold value for the estimated coefficients.

\item[\code{EM\_tol}] 
Convergence tolerance for EM algorithm.

\item[\code{EM\_iter}] 
Maximum number of EM algorithm iterations.

\item[\code{tol}] 
Convergence tolerance for coordinate descent.

\item[\code{maxit}] 
Maximum number of coordinate descent iterations.

\item[\code{fixed\_sigma}] 
A logical. Whether to estimate dispersion parameter sigma.

\end{ldescription}
\end{Arguments}
%
\begin{Details}\relax
Zero-inflated local graphical model
\end{Details}
%
\begin{Value}
An S3 object with the following slots
\begin{ldescription}
\item[\code{beta}] Estimated coefficients vector.
\item[\code{sigma}] Estimated dispersion parameter sigma.
\item[\code{prob}] Estimated probability of structural zero.
\item[\code{pos\_zero}] Indices of zero values.
\item[\code{iteration}] Iteration numbers until convergence.
\item[\code{loglik}] \emph{l\_1}-penalized negative log-likelihood value.
\item[\code{call}] The mathced call.
\end{ldescription}
\end{Value}
%
\begin{Author}\relax
Park Beomjin, 
\end{Author}
%
\begin{References}\relax
Network analysis for Count data with excess zero
\end{References}
%
\begin{Examples}
\begin{ExampleCode}
Not yet.
\end{ExampleCode}
\end{Examples}
\inputencoding{utf8}
\HeaderA{zilgm\_poisson}{Neiborhood selection under the zero-inflated Poisson distribution for Zero-inflated Local Graphical Model}{zilgm.Rul.poisson}
%
\begin{Description}\relax
Zero-inflated Poisson regression with \emph{l\_1}-regularization.
\end{Description}
%
\begin{Usage}
\begin{verbatim}
zilgm_poisson(y, x, lambda, weights = NULL, update_type = c("IRLS", "MM"), penalty.factor = NULL,
              thresh = 1e-6, EM_tol = 1e-6, EM_iter = 500, tol = 1e-6, maxit = 1e+3)
\end{verbatim}
\end{Usage}
%
\begin{Arguments}
\begin{ldescription}
\item[\code{y}] 
A response \emph{y}.

\item[\code{x}] 
A design matrix \emph{x}.

\item[\code{lambda}] 
A regularization parameter to control a level of \emph{l\_1}-penalty.

\item[\code{weights}] 
Weights vector for observations

\item[\code{update\_type}] 
Algorithm for estimating coefficients

\item[\code{penalty.factor}] 
Weights vector for coefficients of each variable.

\item[\code{thresh}] 
Threshold value for the estimated coefficients.

\item[\code{EM\_tol}] 
Convergence tolerance for EM algorithm.

\item[\code{EM\_iter}] 
Maximum number of EM algorithm iterations.

\item[\code{tol}] 
Convergence tolerance for coordinate descent.

\item[\code{maxit}] 
Maximum number of coordinate descent iterations.

\end{ldescription}
\end{Arguments}
%
\begin{Details}\relax
Zero-inflated local graphical model
\end{Details}
%
\begin{Value}
An S3 object with the following slots
\begin{ldescription}
\item[\code{beta}] Estimated coefficients vector.
\item[\code{prob}] Estimated probability of structural zero.
\item[\code{pos\_zero}] Indices of zero values.
\item[\code{iteration}] Iteration numbers until convergence.
\item[\code{loglik}] \emph{l\_1}-penalized negative log-likelihood value.
\item[\code{z}] Estimated latent variable.
\item[\code{call}] The mathced call.
\end{ldescription}
\end{Value}
%
\begin{Author}\relax
Park Beomjin, 
\end{Author}
%
\begin{References}\relax
Network analysis for Count data with excess zero
\end{References}
%
\begin{Examples}
\begin{ExampleCode}
Not yet
\end{ExampleCode}
\end{Examples}
\inputencoding{utf8}
\HeaderA{zilgm\_sim}{Generate simulation data for zilgm}{zilgm.Rul.sim}
%
\begin{Description}\relax
Generate simulation data for zilgm
\end{Description}
%
\begin{Usage}
\begin{verbatim}
zilgm_sum(A, n, p, zlvs, family = c("poisson", "negbin"),
          signal, theta = NULL, noise, is.symm = TRUE)
\end{verbatim}
\end{Usage}
%
\begin{Arguments}
\begin{ldescription}
\item[\code{A}] 
\emph{p} x \emph{p} adjacency matrix

\item[\code{n}] 
The number of observation to generate simulated data.

\item[\code{p}] 
The number of variables or nodes to generate simulated data.

\item[\code{zlvs}] 
The probability of structural zero for each variable.

\item[\code{family}] 
The distribution to draw samples.

\item[\code{signal}] 
The location parameter for distribution.

\item[\code{theta}] 
The dispersion parameter for negative binomial distribution.

\item[\code{noise}] 
The location parameter for noise distribution.

\item[\code{is.symm}] 
A logical. wheter to generate symmetric matrix.

\end{ldescription}
\end{Arguments}
%
\begin{Details}\relax
For simulated data
\end{Details}
%
\begin{Value}
\begin{ldescription}
\item[\code{X}] \emph{n} x \emph{p} simulated data.
\end{ldescription}
\end{Value}
%
\begin{Author}\relax
Park Beomjin,
\end{Author}
%
\begin{References}\relax
Network analysis for Count data with excess zeros
\end{References}
%
\begin{SeeAlso}\relax
\LinkA{zilgm}{zilgm}
\end{SeeAlso}
%
\begin{Examples}
\begin{ExampleCode}
require(ZILGM)
\end{ExampleCode}
\end{Examples}
\printindex{}
\end{document}
